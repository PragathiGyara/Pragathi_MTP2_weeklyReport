\documentclass{article}
\usepackage{graphicx} % Required for inserting images
\usepackage{comment}
\usepackage{hyperref}
\usepackage[a4paper, margin=1in]{geometry} % Sets 1-inch margin on all sides
\usepackage{multicol}
\usepackage{subcaption}
\usepackage{caption}
\usepackage{adjustbox}
\usepackage{tabularx}
\usepackage{makecell}
\usepackage{multirow}
\usepackage{xcolor}
\usepackage{listings}
\usepackage{longtable}
\usepackage{booktabs}
\usepackage{makecell}
\usepackage[
  backend=biber
]{biblatex}

\addbibresource{references.bib}

\lstset{
    basicstyle=\ttfamily\small,
    numbers=left,
    numberstyle=\tiny\color{gray},
    stepnumber=1,
    numbersep=6pt,
    tabsize=2,
    showspaces=false,
    showstringspaces=false,
    breaklines=true,
    keywordstyle=\color{blue},
    commentstyle=\color{gray!90},
    stringstyle=\color{orange!90!black},
    captionpos=b,
    frame=single,
    rulecolor=\color{gray!50},
    backgroundcolor=\color{gray!5}
}


\title{MTP 2 Weekly Report}
\author{Gyara Pragathi}
\date{December 2025}
\begin{document}

\renewcommand{\arraystretch}{1.3}
\setlength{\tabcolsep}{6pt}

\maketitle

\tableofcontents

\newpage
\listoffigures

\newpage
\listoftables

\newpage
\lstlistoflistings

\newpage
\section{WEEK 0 : 16th December 2025 - 19th December 2025} 

Time Spent : 9 - 10 hours

\subsection{Objectives}
\begin{enumerate}
    \item Verify all implemented features
    \item Complete reading Digital Minimalism book
    \item Survey
\end{enumerate}
\subsection{Objective 1 : Testing}
\subsubsection{Frontend}
\begin{enumerate}
    \item Verify all usage metrics are measured accurately
    \begin{itemize}
        \item Screen time
        \item Unlock Count
        \item App launch count
    \end{itemize}
    \item Ensure that usage metrics are consistently represented across the application.
    Currently the usage data is displayed on: 
    \begin{itemize}
        \item Lock Screen
        \item Home Screen
        \item Notification bar
        \item App Drawer
        \item Dashboard
    \end{itemize}
    Test whether: 
    \begin{enumerate}
        \item all components are showing identical values 
        \item are there any inconsistencies due to using shared preferences for storing the usage metrics. 

        Instead, database can be used.
    \end{enumerate}

    \item Test if alarm is triggering on time. Some of the conditions to test under: 
    \begin{itemize}
        \item Low battery
        \item Battery saver mode
        \item Idle state mode
    \end{itemize}

    \item Verify if the alarmManager behaviour works same for all devices. 

    \item How does the alarm behave when phone was switched off during the triggering time.

    \item Check if API calls are efficient.
\end{enumerate}



\subsubsection{Server side}
\begin{enumerate}
    \item Check if celery worker to update points boundary is triggering on time
    \item Check if leaderboard is calculated right?
    \item Backend frontend consistencys
\end{enumerate}


\subsection{Objective 3 : Survey}

\begin{enumerate}
    \item Google Form link: \href{https://docs.google.com/forms/d/1p_3wBV6kxQMTjSB3DH2tA4sDBrUBfmnHJPrCuGDaASA/edit}{Link}
    \item QR Code
\end{enumerate}

\subsection{Observations}
\begin{enumerate}
    \item Created a new account. The baseline data of last 7 days didn't upload (it was uploaded only for previous day). Refer Fig. \ref{fig:Baseline data not uploaded correctly}

    \item The next day usage session data is also considered as baseline. Refer Fig. \ref{fig:Baseline data not uploaded correctly}
\end{enumerate}

\begin{figure}
    \centering
    \includegraphics[width=1\linewidth]{Images/BaselineDataIssue.png}
    \caption{Baseline data not uploaded correctly}
    \label{fig:Baseline data not uploaded correctly}
\end{figure}

\newpage
\section{WEEK 1 : 22nd December 2025 - 26th December 2025 }

Time Spent : 22-23h
\subsection{Objectives}
\begin{enumerate}
    \item Verification of implemented features.
    \item Bug resolution in code to upload last 7 days usage data (baseline data) into the server. \hfill \textbf{[DONE]}
    \item Survey \hfill \textbf{[DONE]}
    \item Analysis of responses from survey 
\end{enumerate}

\begin{comment}
\subsection{New Feature Ideas from the Digital Minimalism book}
\end{comment}

\subsection{Survey}
\begin{enumerate}
    \item Conducted on 24th Dec 2025.
    \item Received 47 responses via google forms.
    \item Observations: 
    \begin{itemize}
        \item Many people were already familiar with digital wellbeing app on android.
    \end{itemize}
\end{enumerate}

\subsection{Uploading Baseline Data}

\begin{enumerate}
    \item The baseline worker is called when the user logins for the first time. The status of baseline data uploaded or not in stored in login pref. If the value is 'false', the baseline data is uploaded, if its 'true' the baseline worker is called.
    
    \item Right now, the baseline worker is not uploading all the 7 days data as intented.
    
    \item Bugs found in the baseline worker:
    \begin{itemize}
        \item Mismatch in session timings. \hfill \textbf{[RESOLVED]}
        
            \textit{Issue:} The day, evening, and night sessions were calculated using the wrong day reference, so some sessions were outside the intended 5 pm–5 pm window. 
            
            \textit{Fix:} The session timings were corrected so that all sessions fall properly within the 5 pm–5 pm day.
        
        \item Anomaly if the baseline worker is called before 5 pm. \hfill \textbf{[RESOLVED]}
        
            \textit{Issue:} When the baseline worker ran before 5 pm, it tried to upload data for a day that was still in progress.
            
            \textit{Fix:} The worker was updated to always upload the last fully completed day by shifting the baseline window when needed.
            
        \item 7 days data not uploaded. Ref \ref{fig:Incomplete Baseline Upload} \hfill \textbf{[RESOLVED]}
\begin{figure} 
            \centering
            \includegraphics[width=1\linewidth]{Images/IncorrectBaselineUpload.png}
            \caption{Incomplete Baseline Upload}
            \label{fig:Incomplete Baseline Upload}
        \end{figure}
                
        \item Error when total session is trying to be uploaded. Ref \ref{fig:Error While Uploading Baseline} \hfill \textbf{[RESOLVED]}

        \textit{Fix:} Temporarily commented out the 
        \texttt{HealthConnectUtils.readStepsBetweenBlocking} since there was no usage.
\begin{figure}
            \centering
            \includegraphics[width=1\linewidth]{Images/ErrorUploadingBaseline.png}
            \caption{Error While Uploading Baseline}
            \label{fig:Error While Uploading Baseline}
        \end{figure}         
    \end{itemize}

    \item Figure \ref{fig:Corrected Baseline Upload} shows the final baseline upload after fixing the above issues. 
\begin{figure}
    \centering
    \includegraphics[width=1\linewidth]{Images/CorrectedBaselineUpload.png}
    \caption{Corrected Baseline Upload}
    \label{fig:Corrected Baseline Upload}
\end{figure}
\end{enumerate}

\subsection{Verification of screen time}
\begin{enumerate}
    \item Different screen time interpretations:
    \begin{itemize}
        \item time screen in ON
        \item time any app is in foreground \textbf{(currently used)}
        \item time user is actively interacting
    \end{itemize}
    \item Methods for verification: 
    \begin{itemize}
        \item Compare with baseline
        \item Controlled experiments
        \item Maintaining correctness
    \end{itemize}
    \item Approach : 
\end{enumerate}

\subsubsection{Compare with baseline}
\begin{enumerate}
    \item Choice of baseline: 
    \begin{itemize}
        \item Primary : Digital Wellbeing
        \item Secondary : popular apps from play store (with high number of downloads)
    \end{itemize}
    \item Why Digital Wellbeing : 
    \begin{itemize}
        \item 
    \end{itemize}
    \item JODO vs Digital Wellbeing. Ref table \ref{tab:Interpretation of screen time - JODO vs Digital Wellbeing}

    \begin{table}
        \centering
        \begin{tabularx}{\textwidth}{l X X}\hline
             &  JODO & Digital Wellbeing\\\hline
             Interpretation of screen time&  Time spent by any app on the foreground expect JODO & Either,
Time spent by any app on the foreground 
(OR)
Time screen is ON\\\hline
             Duration of day&  5 pm to 5 pm & 12 am to 12 am\\ \hline
        \end{tabularx}
        \caption{Interpretation of screen time - JODO vs Digital Wellbeing}
        \label{tab:Interpretation of screen time - JODO vs Digital Wellbeing}
    \end{table}

    \item Approach : 
    \begin{enumerate}
        \item Collect data from digital wellbeing
        \begin{itemize}
            \item Approximate the data to 5 pm to 5 pm. How to approximate : 
            \item Check if the digital wellbeing screen time interpretation matches with JODO, even when JODO is installed in the phone.
        \end{itemize}
        \item Collect data from JODO.
        \item Plot the graph.
        \item Analyze if there any anomalies.
    \end{enumerate}

    \item Difficulties:
    \begin{itemize}
        \item No API to access Digital Wellbeing data.
    \end{itemize}
    
\end{enumerate}

\subsection{Observations}
\begin{enumerate}
    \item Digital Wellbeing app not shown in the app launcher of JODO.
\end{enumerate}

\newpage
\section{WEEK 2: 29th December 2025 - 2nd January 2026}
Time Spent = 6 h + 7 h + 8 h + 2 h + 7 h = 30 h  
\subsection{Verification of Measurement of Usage Metrics}
\begin{enumerate}
    \item Approach : 
    \begin{itemize}
        \item Via our app, allow to note the start and end time of the experiment.
        \item Android exposes 3 ways to note the usage.
        \begin{enumerate}
            \item UsageEvents : logs of events. 
            \item EventStats : aggregate of event types
            \item UsageStats : aggregate of app usage
        \end{enumerate}
        \item So at the end of the experiment - all these methods are accessed and their values are stored in app logs and maybe presented on the app UI.
        \item It is expected to manually note the usage pattern to cross verify if usage metrics are measured correctly.
    \end{itemize}

    \item Implementation : 
    \begin{itemize}
        \item Implemented 3 methods to find the screen time : 
        \begin{enumerate}
            \item Using \texttt{ACTIVITY\_RESUMED} and \texttt{ACTIVITY\_PAUSED} events from \texttt{UsageEvents}. This shall give us the time spent by app in the foreground \cite{Android_UsageEvents}. But summing all sessions - that is, time spent by each app on the foreground, we would get the screen time.

            Also, computing both including and excluding the launcher package from the screen time.

            \begin{lstlisting}[language=Java,
            caption={Calculating screen time using activity resumed and paused flags}]
            long sessionDuration = event.getTimeStamp() - lastTimestamp;

            if(sessionDuration > 0){
                result.appScreenTimeIncludingLauncherMs += sessionDuration;
            }
            if(!pkg.equals(launcherPackage) && !pkg.equals(getPackageName())) {
                result.appScreenTimeExcludingLauncherMs += sessionDuration;      
            } 
            \end{lstlisting}


            \item Using \texttt{SCREEN\_INTERACTIVE} and \texttt{SCREEN\_NON\_INTERACTIVE} events from \texttt{UsageEvents}. This shall give us the time when the display is on and user input is enabled. \cite{Android_UsageEvents}
            \item Aggregating the time spent on foreground using \texttt{UsageStats} - \texttt{getTotalTimeVisible()} and \texttt{getTotalTimeInForeground()} methods. \cite{Android_UsageStats}

            Difference between Visible and Foreground is that, visible is when the screen is on display. Foreground is when user input is enabled for that app.

            So, when we use foreground, split screen case can be handled. That is, when we run on split screen, if visible is used, since screen time is sum of time spent on each app - there is a change that, the screen time exceeds the actual screen on time. But since foreground accounts only for the time when the app has active user interaction, it might be more accurate. 
            \cite{MicrosoftLearn_UsageStats_TimeForeground}
            \cite{MicrosoftLearn_UsageStats_TimeVisible} 

            \textbf{\texttt{getTotalTimeVisible()} is used as fallback if \texttt{getTotalTimeInForeground()} is not available in the android version on the user's phone.}
            
            \begin{lstlisting}[caption={Calculating screen time using UsageStats},label={Calculating screen time using UsageStats}]
                if (android.os.Build.VERSION.SDK_INT >= android.os.Build.VERSION_CODES.Q) {
                    foregroundTimeMs = stats.getTotalTimeVisible();
                } else {
                    foregroundTimeMs = stats.getTotalTimeInForeground();
                }
            \end{lstlisting}

        \end{enumerate}

        \item Implemented 1 method to find unlock count based on the \texttt{KEYGUARD\_SHOWN} flag.

        \item Implemented an interface which allows us to start and end experiment. And then show the results of current experiment results on the interface at the end.

        \item Writing on the measured usage metrics onto the log file. Sample result of the entries made into the log file can be in seen in the figure \ref{fig:Usage Metrics Accuracy Testing Logs}

        \begin{figure}
            \centering
            \includegraphics[width=1\linewidth]{Images/AccuracyTestingLogs.png}
            \caption{Usage Metrics Accuracy Testing Logs}
            \label{fig:Usage Metrics Accuracy Testing Logs}
        \end{figure}
    \end{itemize}

    \item Testing Objectives:
    \begin{enumerate}
        \item Is UsageEvents and UsageStats giving the same data?
        \begin{itemize}
            \item For small continuous usage intervals (5 min - 3 hours)
            \item For large continuous usage intervals (10 hours - 3 days)
            \item For small non-continuous usage intervals
            \item For large non-continuous usage intervals
        \end{itemize}
        \item Which of the 3 methods of screen time is closer to the actual measured screen time?
        \begin{itemize}
            \item For small usage intervals
            \item For large usage intervals
        \end{itemize}
        \item How close is the unlok count measured using UsageEvents is to the actual unlock count.
    \end{enumerate}

    \item Observations:
    \begin{enumerate}
        \item Unlock count measured using UsageEvents also included the cases where the keygurd shown but the phone is not actually unlocked - which matches which the definition of this flag as per the documentation \cite{Android_UsageEvents}. But it doesn't account for the actual unlock count.
    \end{enumerate}
\end{enumerate}


\subsection{Survey Results}
\subsubsection{Participants}
\begin{enumerate}
    \item Number of participants = 47
    \item Age of 78.7\% (i.e.,37) of the participants was under 21. Refer tab.\ref{tab:Survey - Participants Age Distribution}
    
    \begin{table}
        \centering
        \begin{tabular}{|c|c|c|}\hline
             Age Group&  Number of Participants& Percentage\\\hline
             Under 18&  13& 27.7\%\\\hline
             18 - 21&  24& 51.1\%\\\hline
             21 - 24&  9& 19.1\%\\\hline
             24 - 27&  0& 0\\\hline
             27 and above&  1& 2.1\%\\ \hline
 Total& 47&100\%\\\hline
        \end{tabular}
        \caption{Survey - Participants Age Distribution}
\label{tab:Survey - Participants Age Distribution}
    \end{table}

    \item 18 female (38.3\%) and 29 male (61.7\%).
    
\end{enumerate}

\subsubsection{Usage Statistics}
\begin{enumerate}
    \item Based on ranges (that is, when asked to chose their average daily usage in ranges).
    \begin{itemize}
        \item     65.9\% of the participants reported their usage to be between 2-6 hours - with 34\% to be between 2-4 hours and 31.9\% to be between 4-6 hours. Refer fig. \ref{fig:Survey - Smartphone Usage Distribution}
        \item The average usage was 254.7 min/day, i.e., 4.24 hours/day. And the median usage was 5 hours/day. Refer fig. \ref{fig:Survey - Smartphone Usage Summary}
    \end{itemize}
    \item Based on exact figures as entered by the participants (40/47 valid answers)
    \begin{itemize}
        \item Average usage = 261.5 min/day, i.e., 4.358 hours/day.
    \end{itemize}

\begin{figure}
    \centering
    \includegraphics[width=1\linewidth]{Images/SmartphoneUsageDistributio.png}
    \caption{Survey - Smartphone Usage Distribution}
    \label{fig:Survey - Smartphone Usage Distribution}
\end{figure}

\begin{figure}
    \centering
    \includegraphics[width=1\linewidth]{Images/SmartphoneUsageSummary.png}
    \caption{Survey - Smartphone Usage Summary}
    \label{fig:Survey - Smartphone Usage Summary}
\end{figure}
\end{enumerate}

\subsubsection{Participants' Opinions on their Smartphone Usage}
\begin{enumerate}
    \item 53.19\% of the participants either agreed or strongly agreed that they spent more than intended time on their phones. 27.7\% of the participants were neutral about it, while only 19.1\% of the users were satisfied by the time they spend on smartphones. Refer fig. \ref{fig:Spend more time than intended}

    \item 44.6\% reported to be comfortable not doing anything for a while (refer fig. \ref{fig:Comfortable doing nothing}) while only 12.7\% didn't agree, while the others reported neutral.

    \item 57.44\% reported that their smartphones can be distracting at times. Refer fig. \ref{fig:Find phone distracting}

\end{enumerate}


\begin{figure}[htbp]
  \centering

  \begin{subfigure}{0.75\textwidth}
    \centering
    \includegraphics[width=\linewidth]{Images/SurveyUserOpinionGraph1.png}
    \caption{Find phone distracting}
    \label{fig:Find phone distracting}
  \end{subfigure}

  \vspace{0.6cm}

  \begin{subfigure}{0.75\textwidth}
    \centering
    \includegraphics[width=\linewidth]{Images/SurveyUserOpinionGraph2.png}
    \caption{Spend more time than intended}
    \label{fig:Spend more time than intended}
  \end{subfigure}

  \vspace{0.6cm}

  \begin{subfigure}{0.75\textwidth}
    \centering
    \includegraphics[width=\linewidth]{Images/SurveyUserOpinionGraph4.png}
    \caption{Asked if comfortable doing nothing}
    \label{fig:Comfortable doing nothing}
  \end{subfigure}

  \caption{Survey - User Opinions}
  \label{fig:Survey - User Opinions}
\end{figure}

\newpage
\section{WEEK 3: 5th January 2026 - }
Time Spent : 25 hrs
\subsection{Objectives}
\begin{enumerate}
    \item User Study Report
\end{enumerate}

\subsection{Summary of The influence of different intervention measures on improving mobile phone addiction among teenagers or young adults : a systematic review and network meta-analysis}

\subsubsection{Key Terms}
\begin{enumerate}
    \item Traditional meta analysis : Comparing two different studies that is intervention A vs intervention B (or) intervention A vs control group.
    \item Network meta-analysis : 
    \item Cochrane Randomized Trial Risk Tool
    \item Randomized Controlled trails
\end{enumerate}
\subsubsection{Key Points}
\begin{enumerate}
    \item This study performs both a systematic review and network meta analysis of various interventions.
    \item The purpose of the systematic review/ traditional meta analysis was to see if these interventions can do any better than control group.
    \item The purpose of the network meta analysis is to provide a ranking of those interventions based on their effectiveness to reduce (improve) smartphone addiction.
    \item The various interventions considered are : 
    \begin{enumerate}
        \item Aerobic Aerobics
        \item Badminton
        \item Baduanjin
        \item Basketball
        \item Biofeedback
        \item Tai Chi
        \item Table Tennis
        \item Jump rope
        \item Combined Intervention
        \item Mindfulness-Based Theory
        \item Cognitive Therapy
        \item Sanda
        \item Volleyball
        \item Yoga
    \end{enumerate}
    \item Metrics used Smartphone Addiction Scale Shortend Version (SAS-SV) and Mobile Phone Addiction Index (MPAI).
    
\end{enumerate}

\subsection{Summary of Psychosocial interventions for technological addictions}
Reference : \cite{sharma2018psychosocial}

This is a review article published in 2018, discussing theories behind technology addictions and various interventions which are tested so far.
\subsubsection{Key Terms}
\begin{enumerate}
    \item Psychosocial interventions : using psychological or social actions to produce changes in
    \begin{enumerate}
        \item \textbf{Symptoms}, that is, improve physical and mental health
        \item \textbf{Functioning}, that is, performance
        \item \textbf{Well-being}, that is, quality of life and life satisfaction.
    \end{enumerate}
    \item Clinical Trials : \cite{ClinicalTrails_WHO}
    \item Psychotherapy : treatment via communication and interaction. Also called as talk therapy \cite{Psychotheraphy_NIH}.
    \item Maladaptive Cognition : 
\end{enumerate}
\subsubsection{Key Points}
\begin{enumerate}
    \item Psychological and behavioral theories behind technology addictions: 
    \begin{enumerate}
        \item \textbf{Learning Theories :} That is, since technology use gives some kind of positive reinforcement, users are motivated to use more. This is also called as \textbf{operant conditioning}.
        \item \textbf{Reward-deficiency hypothesis :} seeking higher rewards than that from everyday activities.
        \item Impulsivity
        \item \textbf{Cognitive-behavioral models :} Using technology as an escape mechanism from real world problems.
        \item \textbf{Social skills deficiency theories :} drawn towards virtual world because of anxiousness in real world.
    \end{enumerate}
    \item Psychosocial interventions mentioned : 
    \begin{enumerate}
        \item Psychotherapies
        \begin{itemize}
            \item Psychodynamic Therapy
            \item Cognitive-behavioural therapy
            \item Interpersonal psychotherapy
            \item Problem solving therapy
        \end{itemize}
        \item Community-based treatment
        \begin{enumerate}
            \item Assertive community treatment
            \item First episode psychosis interventions
        \end{enumerate}
        \item Vocational Rehabilitation
        \item Peer support services
        \item Integrated care interventions
    \end{enumerate}
    \item The interventions are said to be tested using randomized controlled clinical trials and meta analysis.
    \item Psychotherapy 
    \begin{itemize}
        \item Commonly tested approaches : Cognitive behavioral therapy and Motivational enhancement therapy.
        \item Two methods of testing : Total abstinence and Controlled Use. Controlled Use was prefered.
        \item Cognitive Behavioral Therapy : 
        \item Motivational Enhancement Therapy  
    \end{itemize}

\end{enumerate}

\newpage
\section{WEEK 4 : 12th January 2026 - 16th January 2026}

Time spent = 5 h + 7 h + 7 h (Meeting 1) + 7 h + 7 h = 33 h

\subsection{The influence of different intervention measures on improving mobile phone addiction among teenagers or young adults: a systematic review and network meta-analysis}
Paper Link: \cite{frontiers2025technological} (2025)
\subsubsection{Summary}
This paper conducts a systematic review (traditional meta analysis) and network meta-analysis of various intervention measures to reduce (or improve) mobile phone addiction among teenagers and young adults.
This paper considers 32 studies based on PICOS (Participants,interventions, comparators, outcomes, and study design) methodology (The inclusion criteria is discussed in detailed notes.).
Meta-analysis is done using Rev Man 5.3 software. Network meta-analysis is done using Stata 16.0 software. 
The outcome of the meta-analysis is that all interventions considered perform better than control group (that is, no intervention). 
\textbf{The outcome of the network meta-analysis is that Badminton and Mindfulness-Based Therapy are the best interventions are highest ranked interventions among others.}
\subsubsection{Detailed Notes}
\begin{enumerate}
    \item \textbf{Meta analysis}: Provides aggregate result of various interventions. 
    That is, when there are individual studies on a particular intervention, the studies can be combined statistically to get wider set of participants.
    And hence be more precise.

    This is limited to a single intervention. That is, it tells if one intervention is better than control group or not. Or sometimes offers pretest and posttest analysis - where there is no control group but the observations is made on the same group of people before and after intervention.
    \cite{Meta_Analysis1} \cite{Meta_Analysis2}

    \item \textbf{Network meta-analysis}: In this multiple interventions can be compared at once. It uses transitivity to determine relative ranking between the interventions. 
    That is, according to one intervention A is better than intervention B and according to another study if intervention B is better than intervention C, then network meta analysis says that intervention A is better than intervention C.
    \cite{pmc2017networkmeta} \cite{sciencedirect2024nma} \cite{cochrane2023nma}
    \item Study : 
    \begin{itemize}
        \item Out of 2348 studies from various databases, 32 studies were selected based on the inclusion criteria.
        \item The inclusion criteria (PICOS methodology):
        \begin{enumerate}
            \item Participants : Teenagers or young adults who have mobile phone addiction (based on assessment performed before the start of the individual stuides.)
            \item Interventions : exercise interventions
            \item Comparisons : intervention vs control group.
            \item Outcomes : The addiction levels are measured in scales : SAS-SV (Smartphone Addiction Scale - Shortened Version) and MPAI (Mobile Phone Addiction Index).
            \item Study design : Randomized controlled trials.
        \end{enumerate}
        \item The various interventions (14) considered in the 32 studies are (their description not given in the paper): 
        \begin{enumerate}
            \item Aerobic Aerobics : Refered in the title \href{https://doi.org/10.27441/d.cnki.gyzdu.2022.000378}{Chao Y. Experimental study on the effects of aerobic aerobicsand badminton on
                moderate mobile phone addictionamong secondary vocational school students thesis
                (2022). Yangzhou University.} Unable to access the article. Assuming its aerobics.
            \item Badminton
            \item Baduanjin : form of physical exercise, common in China
            \item Basketball
            \item Biofeedback : monitoring body's involuntary functions like heart rate using monitoring systems. Providing feedback/therapy based on that. \cite{biofeedback}
            \item Tai Chi : form of physical exercise, common in China
            \item Table Tennis
            \item Jump rope
            \item Combined Intervention
            \item Mindfulness-Based Therapy : form of talk therapy
            \item Cognitive Therapy : form of talk therapy
            \item Sanda : form of kickboxing, common in China
            \item Volleyball
            \item Yoga
        \end{enumerate}
        \item Risk of bias assessment was performed using Cochrane Randomized Trial Risk Tool.
        \begin{enumerate}
            \item Selection bias (that is, how the control group and intervention group is choosen)- low risk or unclear risk. 
            \item Performance bias (that is, prior intimidation to the participants about the study can affect the performance during the study) - high risk.
            \item Detection bias (that is, how the outcome is measured. There both SAS-SV and MPAI scales are considered) - high risk.
            \item Incomplete outcome data
            \item Selective reporting
            \item Other sources of bias
        \end{enumerate}
        \item Meta Analysis : meta analysis was performed using a software Rev Man 5.3.
        This software gives SMD for each individual study and also pools SMD across studies.

        \begin{itemize}
            \item Each study has its control group and intervention group. 
            \item The mean is the average of addiction scores (measures via SAS-SV or MPAI - detection bias) for each group.
            \item The pooled standard deviation is calculated using the standard deviations of both groups.
            \item The Standard Mean Difference (SMD) is calculated using: SMD = (Mean of intervention A - Mean of intervention B) / Pooled standard deviation. \cite{standardMeanDifference}
            \item If SMD \(> 0\), then intervention A is better than intervention B.
        \end{itemize}


        Outcome : Pooled SMD (that is, for all studies) = -1.38, 95\% CI (-1.75, -1.01). \textbf{That means, the interventions show a better scores than that of control group.}
        But the heterogeneity is said to be high - that is, around 94\%.

        \item Network meta-analysis : network meta-analysis was performed using the software Stata 16.0.
        The software generates a cummulative ranking probability curve.
        From these curves, the Surface Under the Cumulative Ranking Curve (SUCRA) values were calculated.
        It ranges from 0 to 100, 100 being the best. A SUCRA value is assoicated to each intervention. \cite{sucra_probability}

        Outcome : Badminton had the highest value (93.8) \(> 0\) MBT (77.3) \(> 0\) Badunajin (74.3)..
        \item Limitations as reported:
        \begin{enumerate}
            \item There were relatively few randomized controlled trails involving badminton and MBT considered in this study.
            \item High heterogeneity between the studies. So there is possibility that is can affect the results of the study.
        \end{enumerate}
    \end{itemize}


\end{enumerate}
\subsubsection{Observations}
Positives:
\begin{enumerate}
    \item Attempts to provide ranking of intervention based on various studies - which increases the sample size.
    \item Badminton and Mindfulness-Based Therapy is reported to have a good effect on improving (that is, reducing) mobile phone addiction.
\end{enumerate}
Negatives:
\begin{enumerate}
    \item Out of 32 studies, 31 studies are based in China.
    \item The result of badminton reducing addiction can be biased since "badminton is one of the most participated sports among Chinese
mass sports enthusiasts, with a participation rate as high as 42.6\%."
    \item High risk of performance bias and detection bias.
\end{enumerate}


\subsection{A meta analysis of psychological interventions for Internet/smartphone addiction among adolescents}
Paper Link : \cite{metaanalysis2019internetaddiction} (2019)
\subsubsection{Summary}
This paper does a meta analysis of 6 papers containing psychological interventions such as Cognitive Behavioral Therapy (CBT), sandplay therapy and behaviour based educational programming. 
It explains the reasons for focusing on adolescents and also gives the theories behind compulsive internet usage.
The outcome of the meta analysis is that, these interventions show positive effect in reducing internet addictions. SMD of CBT based studies is -0.60 and SMD of educational programming is -0.76. Though there is no inference made to decide which is better - that is, no comparision is done to say which is better.

\subsubsection{Detailed Notes}
\begin{enumerate}
    \item 'Internet Gaming Disorder' added as a diagnosis in the DSM-5 (Diagnostic and Statistical Manual of Mental Disorders, Fifth Edition).
    \item Reason for focusing on adolescents (10 to 24 year olds) : 
    \begin{enumerate}
        \item Higher risk of psychological disorders like depression, loneliness, social problems and academic issues.
        \item Higher risk of suicidal tendencies
    \end{enumerate}
    \item Reasons for compulsive internet usage:
    \begin{enumerate}
        \item \textbf{Cognitive Behavioral Model} : An individual's thoughts coupled with behaviour leads to abnormal behaviour.
        \item \textbf{Social skills deficit theory} : individual's preference of online over face-to-face social interactions also contributed to additive use of internet. 
        Feeling overwhelmed, personal problems or traumatic events in life can make individuals chose online over face-to-face interactions.
    \end{enumerate}
    \item \textbf{Psychological Intervention} : The purpose of a psychological intervention is to improve:
    \begin{enumerate}
        \item Symptoms : physical and mental health symptoms
        \item Functioning : performance in everyday activities
        \item Well-being : overall quality of life and life satisfaction
    \end{enumerate}
    \item The psychological interventions reviewed in this paper: cognitive based therapy, sandplay therapy, behaviour based educational programming.
    \item \textbf{Cognitive Behavioral Therapy (CBT)} : Based on the cognitive behavioral model. Helps identify the negative or problematic cognition/thought patterns via talk therapy. 
    \item \textbf{Sandplay Therapy} : Where individuals are asked to draw on sand to express their emotions.
    \item \textbf{Behaviour Based Educational Programming} : 
    \item Study:
    \begin{enumerate}
        \item 6 papers (2000 to 2019) were chosen based on the below criteria:
        
        Inclusion criteria:
        \begin{itemize}
            \item Randomized Controlled Trials (RCT) study design. 
            That is, there has to be a control group and intervention group randomly chosen.
            \item Participants are diagnoised for internet addiction.
            \item The purpoe of the study was to find the effects of the intervention.
        \end{itemize}
        Exclusion criteria:
        \begin{itemize}
            \item Single group pre-post comparison study. That is, there was no control group.
            \item Study didn't report mean and standard deviations of the addiction levels/scores.
        \end{itemize}
        \item The interventions include CBT (3 studies), sandplay therapy(1 study) and behaviour based educational programming (2 studies).
        \item Out of 6 studies, 5 were conducted at Asian University and 1 at European University.
        \item A total of 305 adolescents aged 12 to 21 participated in the studies.            \item CBT study :
        \begin{enumerate}
            \item Study 1 (eight session CBT): Initially there was no difference between active and control group at baseline. 
            But after 6 months, both groups showed decrease in scores but only CBT group showed improved time management skills and better emotional, cognitive and behavioral symptoms (stands with the purpose of a psychological intervention).
            \item Study 2 (eight session CBT + medication): Compared to non-CBT (only medication) group, CBT showed no significant decrease in depression symptoms but it showed decrease in time spent on internet gaming.
            \item Study 3 (two-stage intervention CBT - that is, cognitive reconstruction talks + daily logs): Improvement seen in reduced social media addiction and improved mental health and learning performance.
        \end{enumerate}
        \item Sandplay Therapy:
        \begin{enumerate}
            \item Study 1: Showed reduction in time spent on internet.
        \end{enumerate}
        \item Educational programming:
        \begin{enumerate}
            \item The aims of both studies were to improve academic performance, time management skills and reduce internet addiction.
            \item Both studies showed decrease in internet addiction.
        \end{enumerate}
        \item Results : 
        \begin{itemize}
            \item For all studies : SMD = -0.67, 95\% CI = -1.07 to -0.27. 
            \item For CBT : SMD = -0.60, 95\% CI = -1.38 to 0.19.
            \item For educational programming : SMD = -0.76, 95\% CI = -1.38 to -0.13.
            \item It shows that the interventions work better than no intervention at all (than control group).
        \end{itemize}
    \end{enumerate}
\end{enumerate}
\subsubsection{Observations}
Positives
\begin{enumerate}
    \item heterogeneity is less compared to previous paper (current - 64\%, previous -94\%).
\end{enumerate}
Negatives
\begin{enumerate}
    \item Self-reported addiction scores were used.
    \item Only meta analysis performed. That is, based on SMD (since negative), we can say that the interventions in general have positive effect in reducing the internet addiction, but doesn't say which intervention exactly works better. 
\end{enumerate}

\subsection{Psychosocial interventions for technological addictions}
Paper Link : \cite{sharma2018psychosocial} (2018)
\subsubsection{Summary}
This is a review article which talks about the psychosocial interventions, the theories behind technological addictions, and also discusses the interventions Cognitive based therapy and Motivational enhancement therapy in detail.

\subsubsection{Detailed Notes}
\begin{enumerate}
    \item \textbf{Psychosocial interventions} : using both psychological (thoughts, emotions, behaviour) and social factors (people, environment) to produce changes in symptoms, functioning and well-being.
    \item Theories which explain technological addictions: 
    \begin{enumerate}
        \item \textbf{Learning Theories :} That is, since technology use gives some kind of positive reinforcement, users are motivated to use more. This is also called as \textbf{operant conditioning}.
        \item \textbf{Reward-deficiency hypothesis :} seeking higher rewards than that from everyday activities.
        \item \textbf{Impulsivity :} tendency to action without much forethought.
        \item \textbf{Cognitive-behavioral models :} Using technology as an escape mechanism from real world problems.
        \item \textbf{Social skills deficiency theories :} drawn towards virtual world because of anxiousness in real world.
    \end{enumerate}
    \item Broad range of interventions :
    \begin{enumerate}
        \item Psychotherapies
        \begin{itemize}
            \item Psychodynamic Therapy
            \item Cognitive-behavioural therapy
            \item Interpersonal psychotherapy
            \item Problem-solving therapy
        \end{itemize}
        \item Community based treatment
        \begin{itemize}
            \item assertive community treatment
            \item first episode psychosis intervention
        \end{itemize}
        \item Vocational Rehabilitation
        \item Peer support services
        \item Integrated care intervention
    \end{enumerate}
    \item Cognitive behavioural therapy (CBT) and Motivational enhancement therapy (MET) are reported to be most frequently tested.
    \item Psychotherapeutic interventions can be done in two modes : Total abstinence and controlled use of technology. Given present need of smartphones, the model of abstience is recommeded only for the indiduals how failed to do controlled use. 
    \item Cognitive Behavioural Therapy: Based on the cognitive behavioral model. Helps identify the negative or problematic cognition/thought patterns via talk therapy.
    \item 8 techniques suggested for CBT. That is, the therapist should helps the users figure out the following (?) : 
    \begin{enumerate}
        \item Practice the Opposite : disrupt the pattern of usage with a new schedule
        \item External stoppers : have prompts to take break or log off.
        \item Setting goals : set up achievable goals of usage time.
        \item abstinence from certain applications  : Totally not using the apps on which they don't have control.
        \item reminder cards : cards which remind users the cost of internet addiction.
        \item personal inventory : figuring out what activities they are missing out.
        \item support groups 
        \item family therapy
    \end{enumerate}
    \item Motivational Enhancement Therapy : This is also a kind of talk therapy where the therapist does non-confrontational, motivational conversation to figure out a personalized treatment plan and attainable goals. 
    Though this, the goal is also to make users realize their excessive or compulsive use.
\end{enumerate}

\subsection{Usage Metrics Accuracy Testing}
\subsubsection{Case 1}
\begin{enumerate}
    \item Duration : 5 minutes
    \item Details about phone model : Samsung Galaxy M32
    \item Phone Configurations : Screen timeout set to 5 min.
    \item Controlled events : 1 unlock + screen set on launcher after unlock.
    \item Results : 
    
    Refer image \ref{fig:Usage Metrics Accuracy Testing Results - Case 1}
    \begin{figure}
        \centering
        \includegraphics[width=0.25\linewidth]{Images/AccuracyTesting_Experiment14.jpeg}
        \caption{Usage Metrics Accuracy Testing Results - Case 1}
        \label{fig:Usage Metrics Accuracy Testing Results - Case 1}
    \end{figure}

    \begin{enumerate}
        \item Screen Time Method 1
        \begin{itemize}
            \item Including launcher time : 4 min 50 seconds.
            \item Excluding launcher time : 0 seconds.
        \end{itemize}
        \item Screen Time Method 2 : 4 min 56 seconds.
        \item Screen Time Method 3
        \begin{enumerate}
            \item Including launcher time : 2 h 39 min 45 seconds.
            \item Excluding launcher time : 1 h 56 min 57 seconds.
        \end{enumerate}
        \item Unlock Count : 1
    \end{enumerate}
        
    \item Observations : 
    \begin{enumerate}
        \item Method 1 and Method 2 are close to actual screen time. The actual screen time is 5 min 1 sec minus the time taken to lock and unlock the phone.
        \item Method 3 is way off the actual screen time.
        \item Unlock count is correct.
    \end{enumerate}
\end{enumerate}

\subsubsection{Case 2}
\begin{enumerate}
    \item Duration : 10 minutes
    \item Details about phone model : Samsung Galaxy M32
    \item Phone Configurations : Screen timeout set to 5 min
    \item Controlled events list:

    \begin{table}[htbp]
        \centering
        \begin{tabular}{c p{8cm} p{3cm}}
        \toprule
        \textbf{Event No.} & \textbf{Event Description} & \textbf{Time elapsed (hh:mm:ss:ms)} \\
        \midrule
        Event 1 & Move from experiment page (inside app) to launcher screen & 00:00:02.2\\
        Event 2 & Lock the phone & 00:00:04.5 \\
        Event 3 & Phone unlocked and the screen is set to launcher & 00:00:20.1 \\
        Event 4 & App drawer opened & 00:01:44.5 \\
        Event 5 & Other app opened & 00:02:02.5 \\
        Event 6 & The other app closed and screen back to launcher & 00:03:05.0 \\
        Event 7 & App Drawer opened again navigated to another app & 00:03:24.7 \\
        Event 8 & The other app closed and back to launcher & 00:04:54.0 \\
        Event 9 & Opened competition page of Intentor & 00:05:51.9 \\
        Event 10 & Closed intentor and back to launcher & 00:06:50.6 \\
        Event 11 & Lock the phone & 00:07:25.9 \\
        Event 12 & Phone unlocked and the screen is set to launcher & 00:08:10.8 \\
        Event 13 & Opened intentor & 00:09:40.7 \\
        Event 14 & Stopped experiment & 00:09:59.4 \\
        \bottomrule
        \end{tabular}
        \caption{Interaction events captured during experiment - Case 2}
        \label{tab:Interaction events captured during experiment - Case 2}
    \end{table}

    \item Controlled events summary :
    \begin{enumerate}
        \item Time inside JODO application : 58.7 + 18.7 = 77.4 s (01 min 17.4 s)
        
        (Event 9 to Event 10, Event 13 to Event 14)
        \item Time on the launcher : 2.3 + 84.4 + 19.7 + 57.9 + 35.3 + 89.9 = 289.5 s (04 min 49.5 s)
        
        (Event 1 to Event 2, Event 3 to Event 4, Event 6 to Event 7, Event 8 to Event 9, Event 10 to Event 11, Event 12 to Event 13)
        \item Time on app drawer : 18.0 + 5.0 = 23.0 s (00 min 23.0 s)
        
        (Event 4 to Event 5, Event 7)
        \item \label{pt:ActualJODOapp} \textbf{Time spent on JODO application = 77.4 s + 289.5 s + 23.0 s = 389.9 s (06 min 29.9 s)}
        \item \label{pt:ActualOtherApps} \textbf{Time on other applications : 62.5 + 84.3 = 146.8 s (02 min 26.8 s)}
        
        (Event 5 to Event 6, Event 7 to Event 8)
        \item \label{pt:ActualScreenTimeMeasure1} Aggregating time spent on all screens \textbf{(Total screen time)= 77.4 + 289.5 + 23.0 + 146.8 = 536.7 s (08 min 56.7 s)}
        \item Time when locked : 15.6 + 44.9 = 60.5 s (01 min 00.5 s)
        \item Total unlocks : 2
        \item \textbf{Total experiment time = 09 min 59.4 s}
        \item \label{pt:ActualScreenTimeMeasure2} Time when phone was unlocked = Total experiment time - Time when locked = 08 min 58.9 s
    \end{enumerate}

    \item Results :

    Refer image \ref{fig:Usage Metrics Accuracy Testing Results - Case 2}
    \begin{figure}
        \centering
        \includegraphics[width=0.25\linewidth]{Images/AccuracyTesting_Experiment19.png}
        \caption{Usage Metrics Accuracy Testing Results - Case 2}
        \label{fig:Usage Metrics Accuracy Testing Results - Case 2}
    \end{figure}

    \begin{enumerate}
        \item Screen Time Method 1
        \begin{itemize}
            \item \label{pt:ScreenTimeMethod1} Including launcher time (that is, including time spent on intentor/JODO app) : 8 m 49 s
            \item \label{pt:MeasuredOtherApps} Excluding launcher time : 2 m 9 s
        \end{itemize}
        \item Screen Time Method 2 : 9 m 23 s
        \item Screen Time Method 3
        \begin{enumerate}
            \item Including launcher time : 3 h 12 m 19 s
            \item Excluding launcher time : 1 h 59 m 44 s
        \end{enumerate}
        \item Unlock Count : 2
    \end{enumerate}

    \item Observations
    \begin{enumerate}
        % \item There is an approx 2 sec difference in the actual screen time computed as aggregate of time spent on all screens \ref{pt:ActualScreenTimeMeasure1} and computed as time when phone is locked \ref{pt:ActualScreenTimeMeasure2}, that is possibly because of the human error.
        % \item The total screen time (that is, including the launcher time) in method 1 is 8 m 49 s \ref{pt:ScreenTimeMethod1}.
        
        % In this method since time elapsed is calculated as time between \texttt{ACTIVITY\_RESUMED} and \texttt{ACTIVITY\_PAUSED} events, 
        % Event 1 time is not included since \texttt{ACTIVITY\_RESUMED} event has already occurred and the measurement starts from the next \texttt{ACTIVITY\_RESUMED} event which should be Event 3.

        \item Comparison of actual and measured values

\begin{table}[h]
            \centering
            \small
            \begin{tabular}{|p{4cm}|cc|p{2cm}|cc|}
            \hline
            \multirow{2}{*}{\textbf{Metric (Actual)}} 
            & \multicolumn{2}{c|}{\textbf{Method 1}} 
            & \multirow{2}{*}{\makecell{\textbf{Method 2} \\ (9 m 23 s)}} 
            & \multicolumn{2}{c|}{\textbf{Method 3}} \\

            \cline{2-3} \cline{5-6}

            & \makecell{Incl. launcher \\ (08 min 49 s)}
            & \makecell{Excl. launcher \\ (2 m 9 s)}
            & 
            & \makecell{Incl. launcher \\ (3 h 12 m 19 s)}
            & \makecell{Excl. launcher \\ (1 h 59 m 44 s)} \\
            \hline


            Total screen time
            
            (08 min 56.7 s)
            & $-7.7$ s
            & --
            & $+26.3$ s
            & $+3$ h $3$ min $22.3$ s
            & $+39$ min $04.9$ s \\
            \hline

            Time spent on other applications (02 min 26.8 s)
            & --
            & $-17.8$ s
            & --
            & --
            & -- \\
            \hline

            \end{tabular}
            \caption{Difference between measured and actual usage metrics (Case 2)}
            \label{tab:actual-measured-case2}
        \end{table}

        
        \item Method 1 has the least difference of 7.7 s (underestimate) compared to the actual screen time.
        \item Unlock matches
    \end{enumerate}
\end{enumerate}


\subsubsection{Case 3}
\begin{enumerate}
    \item Duration : 2 hours
    \item Details about phone model : Samsung Galaxy M32
    \item Phone Configurations : Screen timeout set to 1 min
    \item Controlled events list:
    
    \begin{longtable}{c p{8cm} p{3cm}}
        \caption{Interaction events captured during experiment - Case 3}
        \label{tab:Interaction events captured during experiment - Case 3} \\

        \toprule
        \textbf{Event No.} & \textbf{Event Description} & \textbf{Time elapsed (hh:mm:ss.ms)} \\
        \midrule
        \endfirsthead

        \toprule
        \textbf{Event No.} & \textbf{Event Description} & \textbf{Time elapsed (hh:mm:ss.ms)} \\
        \midrule
        \endhead

        \midrule
        \multicolumn{3}{r}{\textit{Continued on next page}}
        \\
        \endfoot

        \bottomrule
        \endlastfoot

            Event 1  & Move from experiment page (inside jodo) to launcher & 00:00:10.8 \\
            Event 2  & Phone locked using lock button & 00:00:36.1 \\
            Event 3  & Lock screen (of jodo) shown on double tap & 00:05:10.9 \\
            Event 4  & Lock screen still shown & 00:05:11.0 \\
            Event 5  & Lock screen (of jodo) naturally timed out & 00:06:12.9 \\
            Event 6  & Lock screen (of jodo) shown on double tap & 00:08:36.1 \\
            Event 7  & Lock screen (of phone) shown when swiped & 00:08:54.9 \\
            Event 8  & Lock screen (of phone) naturally timed out & 00:09:55.2 \\
            Event 9  & Lock screen (of jodo) shown on double tap & 00:12:32.0 \\
            Event 10 & Lock screen (of phone) shown when swiped & 00:12:39.9 \\
            Event 11 & Keyguard shown & 00:12:53.0 \\
            Event 12 & Phone unlocked using password, screen set to jodo launcher & 00:13:24.9 \\
            Event 13 & Screen set to app drawer & 00:13:49.3 \\
            Event 14 & App1 opened from app drawer & 00:14:08.0 \\
            Event 15 & App1 closed, screen set to jodo launcher & 00:14:47.9 \\
            Event 16 & Inside jodo app & 00:15:11.1 \\
            Event 17 & Exited jodo app, screen set to jodo launcher & 00:15:48.2 \\
            Event 18 & Screen set to app drawer & 00:15:58.3 \\
            Event 19 & App2 opened from app drawer & 00:16:12.4 \\
            Event 20 & Switch from App 2 to App 1 & 00:18:04.4 \\
            Event 21 & Switch back from App 1 to App 2 & 00:18:12.8 \\
            Event 22 & App 2 closed, screen set to jodo launcher & 00:22:37.0 \\
            Event 23 & Phone locked due to timeout & 00:24:40.1 \\
            Event 24 & Lock screen (of jodo) displayed because of notification & 00:25:10.4 \\
            Event 25 & Lock screen (of phone) shown when swiped & 00:25:35.2 \\
            Event 26 & Keyguard shown & 00:25:45.7 \\
            Event 27 & Phone unlocked using password, screen set to jodo launcher + notification check & 00:26:00.3 \\
            Event 28 & Phone locked due to timeout & 00:27:20.5 \\
            Event 29 & Lock screen (phone) appeared & 00:34:20.3 \\
            Event 30 & Lock screen (phone) disappeared & 00:34:20.4 \\
            Event 31 & Lock screen (of jodo) shown & 00:34:40.8 \\
            Event 32 & Lock screen (of phone) shown when swiped & 00:35:26.6 \\
            Event 33 & Keyguard shown & 00:35:38.6 \\
            Event 34 & Phone unlocked using password, screen set to jodo launcher & 00:36:00.3 \\
            Event 35 & Screen set to app drawer & 00:36:21.1 \\
            Event 36 & App 3 opened from app drawer& 00:36:40.3 \\
            Event 37 & App 3 closed, screen set to jodo launcher & 00:58:00.5 \\
            Event 38 & Phone locked using lock button & 00:58:05.4 \\
            Event 39 & Lock screen (of phone) shown & 01:17:57.1 \\
            Event 40 & Lock screen (of jodo) shown & 01:18:04.5 \\
            Event 41 & Phone unlocked, screen set to jodo launcher & 01:18:11.0 \\
            Event 42 & Phone locked due to timeout & 01:19:12.8 \\
            Event 43 & Phone unlocked using fingerprint, screen set to jodo launcher & 01:21:09.0 \\
            Event 44 & Screen set to app drawer & 01:21:43.4 \\
            Event 45 & App 1 opened from app drawer & 01:22:22.0 \\
            Event 46 & App 1 closed, screen set to jodo launcher & 01:23:42.9 \\
            Event 47 & Phone locked due to timeout & 01:24:42.8 \\
            Event 48 & Phone unlocked using fingerprint, screen set to jodo launcher & 01:39:34.1 \\
            Event 49 & Screen set to app drawer & 01:39:55.0 \\
            Event 50 & App 2 opened from app drawer & 01:40:05.5 \\
            Event 51 & App 2 closed, screen set to jodo launcher & 01:46:04.6 \\
            Event 52 & Phone locked due to timeout & 01:47:08.9 \\
            Event 53 & Receiving call display & 01:51:10.8 \\
            Event 54 & Call display on & 01:51:27.7 \\
            Event 55 & End phone call, lock screen (of phone) shown & 01:52:24.0 \\
            Event 56 & Lock screen (of phone) naturally timed out & 01:53:21.0 \\
            Event 57 & Phone unlocked using fingerprint, screen set to lock screen (of jodo) & 01:56:57.7 \\
            Event 58 & Screen set to jodo launcher when swiped & 01:57:20.6 \\
            Event 59 & Opened competition page and navigated to experiment page & 01:57:52.4 \\
            Event 60 & End experiment & 02:00:00.4 \\

    \end{longtable}

    \item Controlled events summary :
    \begin{enumerate}
        \item Time inside JODO application : 10.8 + 37.1 + 128.0 = 175.9 s (2 min 55.9 s)

        (Event 0 to Event 1, Event 16 to Event 17, Event 59 to Event 60)

        \item Time on the launcher : 25.3 + 24.4 + 23.2 + 10.1 + 123.1 + 80.2 + 20.8 + 4.9 + 34.3 + 59.9 + 64.3 + 31.8 = 502.4 s (8 min 22.4 s)

        (Event 1 to Event 2, Event 12 to Event 13, Event 15 to Event 16, Event 17 to Event 18 and so on..)

        \item Time on app drawer : 18.7 + 14.1 + 19.2 + 38.6 + 10.5 = 101.1 s (1 min 41.1 s)

        \item \textbf{Total time spent on JODO application} : 175.9 + 502.4 + 101.1 = 779.4 s (12 min 59.4 s)

        \item Time spent on other applications : 39.9 + 384.6 + 1280.2 + 80.9 + 359.1 = 2144.7 s (35 min 44.7 s)

        (Event 14 to Event 15, Event 19 to Event 22, Event 36 to Event 37, Event 45 to Event 46, Event 50 to Event 51)

        \item \textbf{Aggregating time spent on all screens (total screen time)} : 502.4 + 101.1 + 2144.7 + 367.9 = 3116.1 s (51 min 56.1 s)

        \item Time on JODO lock screen : 122.0 + 18.8 + 7.9 + 24.8 + 45.8 + 7.4 = 226.7 s (3 min 46.7 s)
        \item Time on phone lock screen : 60.3 + 13.1 + 10.5 + 0.1 + 12.0 + 7.4 + 57.0  = 160.4 s (2 min 40.4 s)
        \item Time on keyguard : 31.9 + 14.6 + 21.7 = 68.2 s (1 min 08.2 s)
        \item Time when locked : 768.8 + 80.2 + 519.8 + 1215.6 + 116.2 + 891.3 + 611.7 = 4203.6 s (70 min 03.6 s)
        \item Total experiment time : 7200.4 s = 2 h 00 min 00.4 s
        \item Time when phone was unlocked : Total experiment time - time when locked = 7200.4 - 4203.6 = 2996.8 s (49 min 56.8 s)
        \item Total unlocks : 7
        
        (Event 12, Event 27, Event 34, Event 41, Event 43, Event 48, Event 57) = 7
    \end{enumerate}


    \item Results :     
    
    Refer image \ref{fig:Usage Metrics Accuracy Testing Results - Case 3}
    \begin{figure}
        \centering
        \includegraphics[width=0.25\linewidth]{Images/AccuracyTesting_Experiment21.png}
        \caption{Usage Metrics Accuracy Testing Results - Case 3}
        \label{fig:Usage Metrics Accuracy Testing Results - Case 3}
    \end{figure}
    
    \begin{enumerate}
        \item Screen Time Method 1
        \begin{itemize}
            \item Including launcher time (that is, including time spent on intentor/JODO app) : 51 min 17 s
            \item Excluding launcher time : 37 m 5 s
        \end{itemize}
        \item Screen Time Method 2 : 56 m 48 s
        \item Screen Time Method 3
        \begin{enumerate}
            \item Including launcher time : 2 h 2 m 29 s
            \item Excluding launcher time : 1 h 31 m 1 s
        \end{enumerate}
        \item Unlock Count : 7
    \end{enumerate}

    \item Observations
    \begin{enumerate}
       \item Comparison of actual and measured values 
       
       \begin{table}[h]
            \centering
            \small
            \begin{tabular}{|p{4cm}|cc|p{2cm}|cc|}
            \hline
            \multirow{2}{*}{\textbf{Metric (Actual)}} 
            & \multicolumn{2}{c|}{\textbf{Method 1}} 
            & \multirow{2}{*}{\makecell{\textbf{Method 2} \\ (56 m 48 s)}} 
            & \multicolumn{2}{c|}{\textbf{Method 3}} \\

            \cline{2-3} \cline{5-6}

            & \makecell{Incl. launcher \\ (51 min 17 s)}
            & \makecell{Excl. launcher \\ (37 m 5 s)}
            & 
            & \makecell{Incl. launcher \\ (2 h 2 m 29 s)}
            & \makecell{Excl. launcher \\ (1 h 31 m 1 s)} \\
            \hline


            Total screen time
            
            (51 min 56.1 s)
            & $-39.1$ s
            & --
            & $+4$ min $51.9$ s
            & $+1$ h $10$ min $32.9$ s
            & $+39$ min $04.9$ s \\
            \hline

            Time spent on other applications (35 min 44.7 s)
            & --
            & $+1$ min $20.3$ s
            & --
            & --
            & -- \\
            \hline

            \end{tabular}
            \caption{Difference between measured and actual usage metrics (Case 3)}
            \label{tab:actual-measured-case3}
        \end{table}

        \item From Method 1, the time spent on JODO application is 14 min 12 s (which is, including launcher time - excluding launcher time).
        The actual time spent on JODO app which is 12 min 59.4 s. There is difference of 72.6 s - underestimation.

        \item From Method 1, the time spent on other applications is 37 m 5 s. The actual time spent on other applications is 35 min 44.7 s. 
        The difference is 80.3 s - overestimation.

        \item When compared the actual total screen time measured with different methods, Method 1 had least difference of 39.1 s underestimation,
        though it might be because of adjustment happened because of the overestimation of time spent on other apps and underestimation of time spent on JODO app, as discussed in the above points.

        When compared with method 2, there was a difference of 4 min 51.9 s overestimate. And compared with Method 3, it was way off, just like in results of 2nd experiment also.
       
        \item The method 2 -- that is, calculating screen time using the \texttt{SCREEN\_INTERACTIVE} and \texttt{SCREEN\_NON\_INTERACTIVE} events --
        takes into the account the time when the screen can recieve any input \cite{Android_UsageEvents}, 
        and by that defination, since the keyguard and lock screen of jodo app are also taking input, they might have also been added to the screen time measured.

        Therefore, adding time the actual time on JODO lock screen and time on keyguard to the actual screen time = 3 min 46.7 s + 1 min 08.2 s + 51 min + 56.1 s = 56 min 51.0 s. 

        And the measured screen time by method 2 is 56 min 48 s. Hence, the difference is of 3 sec overestimation, which is even more accurate than that measured by the method 1.

        \item Unlock count matches.
    \end{enumerate}
\end{enumerate}

\subsubsection{Case 4}
\begin{enumerate}
    \item Duration : 40 minutes
    \item Phone configuration : Screen timeout set to 15 seconds.
    \item Controlled events list : 
    
     \begin{longtable}{c p{8cm} p{3cm}}
        \caption{Interaction events captured during experiment - Case 4}
        \label{tab:Interaction events captured during experiment - Case 4} \\

        \toprule
        \textbf{Event No.} & \textbf{Event Description} & \textbf{Time elapsed (hh:mm:ss.ms)} \\
        \midrule
        \endfirsthead

        \toprule
        \textbf{Event No.} & \textbf{Event Description} & \textbf{Time elapsed (hh:mm:ss.ms)} \\
        \midrule
        \endhead

        \midrule
        \multicolumn{3}{r}{\textit{Continued on next page}}
        \\
        \endfoot

        \bottomrule
        \endlastfoot

        Event 1  & \textbf{Unlock 1} : Unlocked phone using fingerprint & 00:00:45.0 \\
        Event 2  & \textbf{Unlock 2} : Unlocked phone using fingerprint & 00:01:27.1 \\
        Event 3  & \textbf{Unlock 3} : Unlocked phone using fingerprint & 00:01:52.4 \\
        Event 4  & \textbf{Unlock 4} : Unlocked phone using fingerprint & 00:02:11.7 \\
        Event 5  & \textbf{Unlock 5} : Unlocked phone using fingerprint & 00:03:04.7 \\
        Event 6  & Only keyguard shown - phone not unlocked & 00:03:51.0 \\
        Event 7  & Only keyguard shown - phone not unlocked & 00:04:36.5 \\
        Event 8  & Only keyguard shown - phone not unlocked & 00:06:53.7 \\
        Event 9  & Only keyguard shown - phone not unlocked & 00:08:14.4 \\
        Event 10 & Only keyguard shown - phone not unlocked & 00:09:24.4 \\
        Event 11 & \textbf{Unlock 6} : Unlocked phone using password & 00:10:20.9 \\
        Event 12 & \textbf{Unlock 7} : Unlocked phone using password & 00:10:39.1 \\
        Event 13 & \textbf{Unlock 8} : Unlocked phone using password & 00:10:55.2 \\
        Event 14 & \textbf{Unlock 9} : Unlocked phone using password & 00:11:12.3 \\
        Event 15 & \textbf{Unlock 10} : Unlocked phone using password & 00:11:26.8 \\
        Event 16 & \textbf{Unlock 11} : Unlocked phone using fingerprint & 00:12:41.0 \\
        Event 17 & \textbf{Unlock 12} : Unlocked phone using fingerprint & 00:13:11.0 \\
        Event 18 & \textbf{Unlock 13} : Unlocked phone using fingerprint & 00:13:40.4 \\
        Event 19 & Phone was already unlocked & 00:14:00.7 \\
        Event 20 & \textbf{Unlock 14} : Unlocked phone using fingerprint & 00:14:43.7 \\
        Event 21 & \textbf{Unlock 15} : Unlocked phone using fingerprint & 00:16:57.8 \\
        Event 22 & Only keyguard shown - phone not unlocked & 00:17:14.6 \\
        Event 23 & Only keyguard shown - phone not unlocked & 00:17:48.3 \\
        Event 24 & Only keyguard shown - phone not unlocked & 00:18:28.7 \\
        Event 25 & Only keyguard shown - phone not unlocked & 00:19:52.2 \\
        Event 26 & \textbf{Unlock 16} : Unlocked phone using password & 00:19:52.2 \\
        Event 27 & \textbf{Unlock 17} : Unlocked phone using password & 00:20:05.5 \\
        Event 28 & \textbf{Unlock 18} : Unlocked phone using password & 00:20:17.8 \\
        Event 29 & \textbf{Unlock 19} : Unlocked phone using password & 00:20:30.1 \\
        Event 30 & \textbf{Unlock 20} : Unlocked phone using password & 00:21:02.7 \\
        Event 31 & \textbf{Unlock 21} : Unlocked phone using fingerprint & 00:22:30.9 \\
        Event 32 & \textbf{Unlock 22} : Unlocked phone using fingerprint & 00:23:08.9 \\
        Event 33 & \textbf{Unlock 23} : Unlocked phone using fingerprint & 00:23:37.3 \\
        Event 34 & \textbf{Unlock 24} : Unlocked phone using fingerprint & 00:24:02.6 \\
        Event 35 & \textbf{Unlock 25} : Unlocked phone using fingerprint & 00:24:26.4 \\
        Event 36 & Only keyguard shown - phone not unlocked & 00:24:57.9 \\
        Event 37 & Only keyguard shown - phone not unlocked & 00:25:35.6 \\
        Event 38 & Only keyguard shown - phone not unlocked & 00:26:25.6 \\
        Event 39 & Only keyguard shown - phone not unlocked & 00:28:43.5 \\
        Event 40 & Only keyguard shown - phone not unlocked & 00:29.37.9 \\
        Event 41 & \textbf{Unlock 26} : Unlocked phone using password & 00:30:06.0 \\
        Event 42 & \textbf{Unlock 27} : Unlocked phone using password & 00:30:29.8 \\
        Event 43 & \textbf{Unlock 28} : Unlocked phone using password & 00:30:50.2 \\
        Event 44 & \textbf{Unlock 29} : Unlocked phone using password & 00:31:09.1 \\
        Event 45 & \textbf{Unlock 30} : Unlocked phone using password & 00:31:26.7 \\
        Event 46 & \textbf{Unlock 31} : Unlocked phone using fingerprint & 00:32:52.1 \\
        Event 47 & \textbf{Unlock 32} : Unlocked phone using fingerprint & 00:34:09.8 \\
        Event 48 & \textbf{Unlock 33} : Unlocked phone using fingerprint & 00:35:22.7 \\
        Event 49 & \textbf{Unlock 34} : Unlocked phone using fingerprint & 00:36:26.0 \\
        Event 50 & \textbf{Unlock 35} : Unlocked phone using fingerprint & 00:40:03.2 \\



    \end{longtable}

    \item Controlled events summary :
    \begin{enumerate}
        \item Number of unlocks using fingerprint : 20
        \item Number of unlocks using password : 15
        \item Number of events where keyguard was opened, but phone was not unlocked : 15
        \item \textbf{Total number of unlocks : 35.}
    \end{enumerate}
    
    \item Results:
    
    Refer image : \ref{fig:Usage Metrics Accuracy Testing Results - Case 4}
    \item The total unlocks recorded is 34.
    
    \begin{figure}[t]
        \centering
        \begin{subfigure}{0.25\textwidth}
            \centering
            \includegraphics[width=\linewidth]{Images/AccuracyTesting_Experiment27a.png}
            \caption{Results}
            \label{fig:Case 4 a}
        \end{subfigure}
        \begin{subfigure}{0.25\textwidth}
            \centering
            \includegraphics[width=\linewidth]{Images/AccuracyTesting_Experiment27b.png}
            \caption{results - continued}
            \label{fig:Case 4 b}
        \end{subfigure}
        \caption{Usage Metrics Accuracy Testing Results - Case 4}
        \label{fig:Usage Metrics Accuracy Testing Results - Case 4}
    \end{figure}

    \item Observations: 
    \begin{enumerate}
        \item The timestamps of the experiment recorded didn't match with the external timer - assuming its because the timestamp recorded in the experiment is that of when the keyguard corresponding to the unlock is displayed and not that of the acutal unlock.
        \item Events where the keyguard is shown, but the phone is not unlocked are successfully ignored.
        \item Both unlocks - that is, using password and fingerprints are captured.
        \item The recorded number of unlocks (34) is 1 less than the actual unlock count (35), because the last unlock is not recorded.
    \end{enumerate}

\subsection{Implementaiton to log unlock count}
\begin{enumerate}
    \item In the earlier implementation, when the unlock event, that is, the \texttt{KEYGUARD\_SHOWN} came across in the event logs, the unlock count was incremented.
    \item Now modified to maintain list of unlock events. 
    \item At the end of experiment, the list is parsed and the unlock event along with timestamp is displayed on the UI \ref{fig:Usage Metrics Accuracy Testing Results - Case 4} 
    and also this is added into the app logs for persitency.
\end{enumerate}
    
\end{enumerate}





\newpage
\section{WEEK 5 : 19th January 2026 - }

Time spent = 5 h + 8 h + 

\subsection{SAS-SV and MPAI}
\subsubsection{SAS-SV} 
Reference : 
\begin{enumerate}
    \item SAS (Smartphone Addiction Scale)
    \begin{itemize}
        \item Consists of 6 factors and 33 items with 6 point Likert scale (1: "strongly disagree and 6: "strongly agree").
        \item The 6 factors are : daily-life distubrance, positive anticipation, withdrawal, cyberspace-oriented relationship, overuse and tolernace. 
        \item It is one of the widely used scale for self-reporting of problematic smartphone use.
    \end{itemize}
    \item There is a study \cite{SAS_SV} conducted in South Korea on age group of 14-15 for SAS-SV (the shortend version). 
    \item SAS-SV significantly correlated with SAS, SAPS (Smartphone Addiction Proneness Scale) a
    nd KS-Scale (Korean self-reporting internet addiction scale). 
    \item SAS-SV (Smartphone Addiction Scale - Shortend Version). The questions are per SAS-SV are:
    \begin{enumerate}
        \item Missing planned work due to smartphone use
        \item Having a hard time concentrating in class, while doing assignments, or while working due to smartphone use
        \item Feeling pain in the wrists or at the back of the neck while using a smartphone
        \item Won't be able to stand not having a smartphone
        \item Feeling impatient and fretful when I am not holding my smartphone
        \item Having my smartphone in my mind even when I am not using it
        \item I will never give up using my smartphone even when my daily life is already greatly affected by it.
        \item Constantly checking my smartphone so as not to miss conversations between other people on Twitter or Facebook
        \item Using my smartphone longer than I had intented
        \item The people around me tell me that I use my smartphone too much
    \end{enumerate}
    \item The cutoff suggested by this study is 31 for boys and 33 for girls. That is, if the score is above this, it is considered to be problematic smartphone use.
    \item There is another study \cite{SAS_SV_India} conducted in India on 434 participants (Mage = 25.4; SDage = 2.6; 58.8\% females), 
    which validated the use the SAS-SV to study the problematic smartphone usage on young adults of India.

    This study also uses the 6 point Likert scale.

    \item Also, this study does some wording changes to the SAS-SV proposed by South Korean paper.
    \begin{enumerate}
        \item I miss planned work due to smartphone use
        \item I have difficulties concentrating in class, while doing assignmnets, or while working due to smartphone use.
        \item I experience discomfort in my wrists, thumbs, eyes, or at the back of the neck due to smartphone use.
        \item I will not to able to tolerate not having a smartphone
        \item I fell impatient or irritable when I am not holding my smartphone.
        \item I keep thinking of my smartphone even when I am not using it
        \item I may not be able to give up my smartphone even though my daily life is affected by it.
        \item I constantly check my smartphone so as not to miss any conversations on social media (WhatsApp, Instagram, Facebook, Reddit, Twitter, etc)
        \item I use my smartphone longer than I intended or planned
        \item My family or friends have told me that I use my smartphone too much. 
    \end{enumerate}
    \item Though this paper doesn't give any new cutoff for problematic smartphone usage.
    
\end{enumerate}

\subsubsection{MPAI}

\subsection{Mindfulness Based Therapy}

\subsection{Testing if Alarm is triggering on time}
\begin{enumerate}
    \item Current implementation: 
    \begin{itemize}
        \item We are using Alarm Manager to schedule and receive the alarms.
        \item From our application, we need to schedule the alarm by mentioning the time and configuring it to run at regular intervals.
        \item We are using the following methods to schedule the alarm \cite{Schedule_Alarm} \cite{AlarmManager_Medium} (refer listing \ref{label: Methods to schedule Alarm}): 

        \begin{enumerate}
            \item \texttt{set} : To save up the battery, Android might delay the alarms and trigger them all at once.
            \item \texttt{setExact} : Doesn't delay to trigger the alarms in batch, but might be delayed if the phone is in Doze mode (that is, idle for a long time) or battery saving mode. 
            \item \texttt{setExactAndAllowWhileIdle} : Is said to work even if the phone is in doze mode or battery saving mode. Available from Andorid M onwards. 
        \end{enumerate}
        \item \texttt{setExact} is used as a fallback to \texttt{setExactAndAllowWhileIdle} if the user doesn't have required android version. And \texttt{set} is used as fallback to \texttt{setExact}
    
 \begin{lstlisting}[caption={Methods used to schedule alarm},label={label: Methods to schedule Alarm}]
    private void setExactAlarm(AlarmManager alarmManager, long triggerAtMillis, PendingIntent pendingIntent) {
    if (Build.VERSION.SDK_INT >= Build.VERSION_CODES.M) {
            alarmManager.setExactAndAllowWhileIdle(AlarmManager.RTC_WAKEUP, triggerAtMillis, pendingIntent);
    } else if (Build.VERSION.SDK_INT >= Build.VERSION_CODES.KITKAT) {
            alarmManager.setExact(AlarmManager.RTC_WAKEUP, triggerAtMillis, pendingIntent);
    } else {
            alarmManager.set(AlarmManager.RTC_WAKEUP, triggerAtMillis, pendingIntent);
    }
 }
 \end{lstlisting}

        \item Once the alarm is scheduled, the request is sent to AlarmManager, which stores all the alarm requests and monitors time.
        
        If its time, the alarm is triggered - that is, it informs the respective applciation and that application inturn calls the worker which executes the task as per the alarm request code.
    \end{itemize}

    \item Modifications made : 
    \begin{enumerate}
        \item Before logging the phone status, wanted to verify if its right. 
        So created the UI where we can click to get the following information about the phone.
        \begin{itemize}
            \item Clock time
            \item Battery (can be measured using \texttt{BatteryManager} API)
            \item Charging (can be measured using \texttt{BatteryManager} API)
            \item Doze Mode (can be measured using \texttt{PowerManager} API)
            \item Power Saver (can be measured using \texttt{PowerManager} API)
            \item Screen ON (can be measured using \texttt{PowerManager} API)
        \end{itemize}
    \end{enumerate}

\end{enumerate}


 


\newpage
\printbibliography
\end{document}